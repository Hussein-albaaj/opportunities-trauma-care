% Options for packages loaded elsewhere
\PassOptionsToPackage{unicode}{hyperref}
\PassOptionsToPackage{hyphens}{url}
%
\documentclass[
]{article}
\title{Study plan}
\author{}
\date{\vspace{-2.5em}}

\usepackage{amsmath,amssymb}
\usepackage{lmodern}
\usepackage{iftex}
\ifPDFTeX
  \usepackage[T1]{fontenc}
  \usepackage[utf8]{inputenc}
  \usepackage{textcomp} % provide euro and other symbols
\else % if luatex or xetex
  \usepackage{unicode-math}
  \defaultfontfeatures{Scale=MatchLowercase}
  \defaultfontfeatures[\rmfamily]{Ligatures=TeX,Scale=1}
\fi
% Use upquote if available, for straight quotes in verbatim environments
\IfFileExists{upquote.sty}{\usepackage{upquote}}{}
\IfFileExists{microtype.sty}{% use microtype if available
  \usepackage[]{microtype}
  \UseMicrotypeSet[protrusion]{basicmath} % disable protrusion for tt fonts
}{}
\makeatletter
\@ifundefined{KOMAClassName}{% if non-KOMA class
  \IfFileExists{parskip.sty}{%
    \usepackage{parskip}
  }{% else
    \setlength{\parindent}{0pt}
    \setlength{\parskip}{6pt plus 2pt minus 1pt}}
}{% if KOMA class
  \KOMAoptions{parskip=half}}
\makeatother
\usepackage{xcolor}
\IfFileExists{xurl.sty}{\usepackage{xurl}}{} % add URL line breaks if available
\IfFileExists{bookmark.sty}{\usepackage{bookmark}}{\usepackage{hyperref}}
\hypersetup{
  pdftitle={Study plan},
  hidelinks,
  pdfcreator={LaTeX via pandoc}}
\urlstyle{same} % disable monospaced font for URLs
\usepackage[margin=1in]{geometry}
\usepackage{graphicx}
\makeatletter
\def\maxwidth{\ifdim\Gin@nat@width>\linewidth\linewidth\else\Gin@nat@width\fi}
\def\maxheight{\ifdim\Gin@nat@height>\textheight\textheight\else\Gin@nat@height\fi}
\makeatother
% Scale images if necessary, so that they will not overflow the page
% margins by default, and it is still possible to overwrite the defaults
% using explicit options in \includegraphics[width, height, ...]{}
\setkeys{Gin}{width=\maxwidth,height=\maxheight,keepaspectratio}
% Set default figure placement to htbp
\makeatletter
\def\fps@figure{htbp}
\makeatother
\setlength{\emergencystretch}{3em} % prevent overfull lines
\providecommand{\tightlist}{%
  \setlength{\itemsep}{0pt}\setlength{\parskip}{0pt}}
\setcounter{secnumdepth}{-\maxdimen} % remove section numbering
\newlength{\cslhangindent}
\setlength{\cslhangindent}{1.5em}
\newlength{\csllabelwidth}
\setlength{\csllabelwidth}{3em}
\newlength{\cslentryspacingunit} % times entry-spacing
\setlength{\cslentryspacingunit}{\parskip}
\newenvironment{CSLReferences}[2] % #1 hanging-ident, #2 entry spacing
 {% don't indent paragraphs
  \setlength{\parindent}{0pt}
  % turn on hanging indent if param 1 is 1
  \ifodd #1
  \let\oldpar\par
  \def\par{\hangindent=\cslhangindent\oldpar}
  \fi
  % set entry spacing
  \setlength{\parskip}{#2\cslentryspacingunit}
 }%
 {}
\usepackage{calc}
\newcommand{\CSLBlock}[1]{#1\hfill\break}
\newcommand{\CSLLeftMargin}[1]{\parbox[t]{\csllabelwidth}{#1}}
\newcommand{\CSLRightInline}[1]{\parbox[t]{\linewidth - \csllabelwidth}{#1}\break}
\newcommand{\CSLIndent}[1]{\hspace{\cslhangindent}#1}
\ifLuaTeX
  \usepackage{selnolig}  % disable illegal ligatures
\fi

\begin{document}
\maketitle

\hypertarget{introduction}{%
\section{Introduction}\label{introduction}}

\textless\textless\textless\textless\textless\textless\textless{} HEAD
Defined as physical injury and the body´s consequential response,
trauma, is one of the leading causes of mortality and morbidity in all
age groups and the leading cause of mortality in people below the age of
44 {[}\protect\hyperlink{ref-David2021}{1}{]}
{[}\protect\hyperlink{ref-champion1990}{2}{]}. Approximately 5.8 million
people die each year due to trauma. Motor vehicle crashes alone stand
for more than one million deaths and roughly between 20 and 50 million
injuries annually {[}\protect\hyperlink{ref-Committee2013}{3}{]}. With
an average of more than 7 days in the hospital each stay
{[}\protect\hyperlink{ref-champion1990}{2}{]}, it is one of the most
prevalent reasons for admission. In 2020 road traffic accidents were
third in disease burden worldwide, measured by Disability-Adjusted Life
Years (DALY), a term used to describe the impact of health problems and
to measure the significance of improvement in medical care
{[}\protect\hyperlink{ref-Murray1996}{4}{]},{[}\protect\hyperlink{ref-Haagsma2015}{5}{]}.

In many cases the outcome is mostly dependent on the quality of care
acquired {[}\protect\hyperlink{ref-Dogrul2020}{6}{]}. In a teaching
hospital in Tehran, reviews of all trauma cases in 1 year displayed
inappropriate care in 45\% of all deaths and implied that approximately
26\% of all trauma deaths were preventable
{[}\protect\hyperlink{ref-Zafarghandi2003}{7}{]}. Preventable death
panels aim to decrease the amount of preventable deaths by investigating
common factors between the cases
{[}\protect\hyperlink{ref-Jung2019}{8}{]}. Airway management, inadequate
chest compression, inadequate blood or fluid supply are some factors
that previously were found in need of improvement in trauma care
{[}\protect\hyperlink{ref-Zafarghandi2003}{7}{]},{[}\protect\hyperlink{ref-Maio1996}{9}{]}.
Such advances in trauma care are the leading cause of the decreased
amount of preventable deaths according to the American College of
Surgeons Committee {[}\protect\hyperlink{ref-Committee2013}{3}{]}.

It has always been of great importance to investigate factors that can
be improved in trauma care and different methods are used to do so. The
golden standard being mortality and morbidity (M\&M) conferences, also
known as deaths and complications reviews. Some cases get selected for
review to further investigate the reason behind the mortality or
morbidity {[}\protect\hyperlink{ref-WHO2009}{10}{]}. The results should
be used to improve the trauma care and to decrease the amount of
preventable errors. Although the selection of patients is a vital part
of these conferences, it remains a process that requires a great amount
of resources and is still complicated to this day due to the lack of
studies on the subject.

Despite the evidence supporting the use of predefined models for case
selection, there is limited data on specific factors associated with
opportunities for improvement in trauma care
{[}\protect\hyperlink{ref-Slater2020}{11}{]}. Therefore methods such as
audit filters are adopted in trauma quality improvement programs. Audit
filters are predefined factors used in the selection of cases for review
and represent an unfavorable alternation, proclaimed leading to a
disadvantageous outcome
{[}\protect\hyperlink{ref-WHO2009}{10}{]},{[}\protect\hyperlink{ref-Evans2009}{12}{]}.
Set audit filters are for instance systolic blood pressure under 90,
Glasgow Coma Scale less than 9 and not intubated, time to acute
intervention more than 60 minutes and a few more. A systematic review of
audit filters in 2009 found no studies meeting set criteria determining
the effectiveness of set audit filters
{[}\protect\hyperlink{ref-Evans2009}{12}{]}.

To this day trauma quality improvement programs rely on set filters for
the selection of cases for M\&M conferences. Some studies have found no
major opportunities for improvement in currently used audit filters and
believe further advancements are essential
{[}\protect\hyperlink{ref-Cryer1996}{13}{]},{[}\protect\hyperlink{ref-Copes1995}{14}{]}.
We hypothesize that the selection of cases for multidisciplinary
mortality and morbidity reviews can be refined by the usage of other
factors. This can be achieved by the examination of many other factors
registered in a database. We aim therefore to find factors associated
with improvement in trauma care for later development of models trained
to identify cases with capacity for improvement.

\hypertarget{methods}{%
\section{Methods}\label{methods}}

\hypertarget{study-design}{%
\subsubsection{Study design}\label{study-design}}

Using a trauma registry and a trauma care quality database from
Karolinska University Hospital in Solna, Sweden, will we create a
registry based cohort study. Only about ten percent of the patients
included in the trauma registry are also included in the trauma care
quality database, since it is a subset of the registry for patients
selected for review. Comparison of the two databases will be made to
obtain factors that are correlated to improvement in trauma care. \#\#\#
Setting

As previously mentioned trauma care is one of the most prevalent causes
for admission and dedicated trauma teams are of utter importance to
provide the best care possible. Therefore trauma patients are directly
transported to a hospital with such competence in Sweden. Karolinska
University Hospital provide care for all trauma patients in Stockholm,
and include trauma patients in two different databases. 21 000 patients
are included in the trauma registry and 2 200 of which are also included
in the trauma care quality database, those patients were selected for
review. All the patients included in the trauma registry are treated
between 2012 and 2021. The trauma care quality database include
opportunities for improvement as the main outcome and is defined as yes
or no, identified by multidisciplinary review boards. Since our aim is
to identify factors associated with improvement in trauma care, the
database used will therefore have some factors that can be further
examined. The information registered in the database is divided into
sections. First of all the transportation is included, the time taken
from reported trauma until the patient arrives at the hospital.
Demographics, including age and gender. Vital signs such as blood
pressure, respiratory rate, saturation, pulse, temperature and also
Glasgow coma scale upon arrival
{[}\protect\hyperlink{ref-Swetrau2020}{15}{]}. Glasgow coma scale is a
assessment used to determine the conscious level of the patient and if
there is a probability of neurological damage
{[}\protect\hyperlink{ref-Teasdale2014}{16}{]}. The types of injuries
acquired and the mechanism of the trauma. Procedures performed on the
patient are also included {[}\protect\hyperlink{ref-Swetrau2020}{15}{]}.

\hypertarget{participants}{%
\subsubsection{Participants}\label{participants}}

The main criteria for the patients is that they are included in the
trauma registry and were treated between 2012 and 2021. The data is
retrieved from Karolinska University Hospital. Since our aim is to only
identify factors associated with opportunities for improvement in adult
trauma patients, younger patients will be excluded. The age cut-off is
set at ≥15, meaning only patients 15 years old and older and included.
It is set at 15 since patients at that age are treated by the adult
trauma team in Sweden. Many guidelines referred to by the American
College of Surgeons Committee also include the same age criteria
{[}\protect\hyperlink{ref-Committee2013}{3}{]}.

\hypertarget{variables}{%
\subsubsection{Variables}\label{variables}}

\textbf{Study outcome}

The two different databases will be compared to determine factors
associated with capacity for improvement. Both unadjusted and adjusted
factors will be estimated. Independent or adjusted factors meaning there
are still opportunities for improvement after adjustment for other
factors. The predictors used in the link between the two databases are
described in the next paragraph. In the trauma care quality database the
variable opportunity for improvement is either yes or no. Our aim is to
compare those cases with and without opportunities for improvement to
locate factors often more associated with opportunities for improvement.
The results can later be used to develop models trained to identify
cases with those factors.

\textbf{Predictors}

As previously mentioned some predictors will be used to link the two
databases. Those predictors include: Demographics, including gender and
age, vital signs, Glasgow coma scale, injuries acquired and procedures
performed upon patient. All of these factors are already included in the
trauma registry.

\hypertarget{data-sources-and-measurements}{%
\subsubsection{Data sources and
measurements}\label{data-sources-and-measurements}}

All of the data is previously collected in the databases. Registered by
Karolinska Univeristy Hospital personnel. Age and gender are collected
from the patients personal number. Vital sign are measured by staff,
usually nurses in Sweden, and uploaded to the database. Wherever if
there are any false measurements is difficult to say, however the same
vitals signs and Glasgow coma scale value are used in the treatment of
the patient. The other predictors are registered by staff to the
databases.

\hypertarget{bias}{%
\subsubsection{Bias}\label{bias}}

Since this is not a blinded study it is of utter importance to
acknowledge the risk of bias. We will however use simulated data to
develop the analysis model and further on implement it on the data
collected from the databases. This is done to lower the risk of bias.
The analysis model will be inspected by a professional analyst after
development on the simulated data.

\hypertarget{study-size}{%
\subsubsection{Study size}\label{study-size}}

The patients included are all from the databases previously mentioned.
The trauma registry includes around 21 000 patients, all treated between
2012 and 2021. The trauma care quality database is a subset of the
registry and includes about 2 200 of the patients from the registry. The
cases in the trauma care quality database have been reviewed by
professionals and our outcome of opportunities for improvement is
included.

\hypertarget{quantitive-variables}{%
\subsubsection{Quantitive variables}\label{quantitive-variables}}

Such variable include vital signs. Pulse, Blood pressure, temperature
and respiratory rate. Glasgow coma scale as previously described. All
registered to the databases.

\hypertarget{statistical-methods}{%
\subsubsection{Statistical methods}\label{statistical-methods}}

\hypertarget{references}{%
\section*{References}\label{references}}
\addcontentsline{toc}{section}{References}

\hypertarget{refs}{}
\begin{CSLReferences}{0}{0}
\leavevmode\vadjust pre{\hypertarget{ref-David2021}{}}%
\CSLLeftMargin{1 }
\CSLRightInline{David SD, Roy N, Solomon H, Lundborg CS, Wärnberg MG.
Measuring post-discharge socioeconomic and quality of life outcomes in
trauma patients: A scoping review. \emph{Journal of Patient-Reported
Outcomes} 2021;\textbf{5}.
doi:\href{https://doi.org/10.1186/s41687-021-00346-6}{10.1186/s41687-021-00346-6}}

\leavevmode\vadjust pre{\hypertarget{ref-champion1990}{}}%
\CSLLeftMargin{2 }
\CSLRightInline{Champion HR, Copes WS, Sacco WJ, Lawnick MM, Keast SL,
FREY CF. The major trauma outcome study: Establishing national norms for
trauma care. \emph{Journal of Trauma and Acute Care Surgery}
1990;\textbf{30}:1356--65.}

\leavevmode\vadjust pre{\hypertarget{ref-Committee2013}{}}%
\CSLLeftMargin{3 }
\CSLRightInline{Advanced trauma life support ({ATLS}{\textregistered}).
\emph{Journal of Trauma and Acute Care Surgery}
2013;\textbf{74}:1363--6.}

\leavevmode\vadjust pre{\hypertarget{ref-Murray1996}{}}%
\CSLLeftMargin{4 }
\CSLRightInline{Murray CJ, Lopez AD, Organization WH, \emph{et al.}
\emph{The global burden of disease: A comprehensive assessment of
mortality and disability from diseases, injuries, and risk factors in
1990 and projected to 2020: summary}. World Health Organization 1996. }

\leavevmode\vadjust pre{\hypertarget{ref-Haagsma2015}{}}%
\CSLLeftMargin{5 }
\CSLRightInline{al. JAH et. The global burden of injury: Incidence,
mortality, disability-adjusted life years and time trends from the
global burden of disease study 2013. \emph{Injury Prevention}
2015;\textbf{22}:3--18.}

\leavevmode\vadjust pre{\hypertarget{ref-Dogrul2020}{}}%
\CSLLeftMargin{6 }
\CSLRightInline{Dogrul BN, Kiliccalan I, Asci ES, Peker SC. Blunt trauma
related chest wall and pulmonary injuries: An overview. \emph{Chinese
Journal of Traumatology} 2020;\textbf{23}:125--38.}

\leavevmode\vadjust pre{\hypertarget{ref-Zafarghandi2003}{}}%
\CSLLeftMargin{7 }
\CSLRightInline{Zafarghandi M-R, Modaghegh M-HS, Roudsari BS.
Preventable trauma death in tehran: An estimate of trauma care quality
in teaching hospitals. \emph{The Journal of Trauma: Injury, Infection,
and Critical Care} 2003;\textbf{55}:459--65.}

\leavevmode\vadjust pre{\hypertarget{ref-Jung2019}{}}%
\CSLLeftMargin{8 }
\CSLRightInline{Jung K, Kim I, Park SK, Cho H, Park CY, Yun J-H,
\emph{et al.} Preventable trauma death rate after establishing a
national trauma system in korea. \emph{Journal of Korean Medical
Science} 2019;\textbf{34}.
doi:\href{https://doi.org/10.3346/jkms.2019.34.e65}{10.3346/jkms.2019.34.e65}}

\leavevmode\vadjust pre{\hypertarget{ref-Maio1996}{}}%
\CSLLeftMargin{9 }
\CSLRightInline{Maio RF, Burney RE, Gregor MA, Baranski MG. A study of
preventable trauma mortality in rural michigan. \emph{The Journal of
Trauma: Injury, Infection, and Critical Care} 1996;\textbf{41}:83--90.}

\leavevmode\vadjust pre{\hypertarget{ref-WHO2009}{}}%
\CSLLeftMargin{10 }
\CSLRightInline{World health organization, guidelines for trauma quality
improvement programmes {[}internet{]}. World health organization.
\url{https://apps.who.int/iris/bitstream/handle/10665/44061/9789241597746\%7B/_\%7Deng.pdf;jsessionid=E90A5736D4F3786CCBAA19E19E4EEF5F?sequence=1}
(accessed 2022).}

\leavevmode\vadjust pre{\hypertarget{ref-Slater2020}{}}%
\CSLLeftMargin{11 }
\CSLRightInline{Slater N, Sekhon P, Bradley N, Shariff F, Bedford J,
Wong H, \emph{et al.} Morbidity and mortality conferences in general
surgery: A narrative systematic review. \emph{Canadian Journal of
Surgery} 2020;\textbf{63}:E211--22.}

\leavevmode\vadjust pre{\hypertarget{ref-Evans2009}{}}%
\CSLLeftMargin{12 }
\CSLRightInline{Evans C, Howes D, Pickett W, Dagnone L. Audit filters
for improving processes of care and clinical outcomes in trauma systems.
\emph{Cochrane Database of Systematic Reviews} Published Online First:
October 2009.
doi:\href{https://doi.org/10.1002/14651858.cd007590.pub2}{10.1002/14651858.cd007590.pub2}}

\leavevmode\vadjust pre{\hypertarget{ref-Cryer1996}{}}%
\CSLLeftMargin{13 }
\CSLRightInline{Cryer HG, Hiatt JR, Fleming AW, Gruen JP, Sterling J.
Continuous use of standard process audit filters has limited value in an
established trauma system. \emph{The Journal of Trauma: Injury,
Infection, and Critical Care} 1996;\textbf{41}:389--95.}

\leavevmode\vadjust pre{\hypertarget{ref-Copes1995}{}}%
\CSLLeftMargin{14 }
\CSLRightInline{Copes WS, Staz CF, Konvolinka CW, Sacco WJ. American
college of surgeons audit filters. \emph{The Journal of Trauma: Injury,
Infection, and Critical Care} 1995;\textbf{38}:432--8.}

\leavevmode\vadjust pre{\hypertarget{ref-Swetrau2020}{}}%
\CSLLeftMargin{15 }
\CSLRightInline{Swetrau, årsrapport SweTrau 2019 {[}internet{]}. Annual
report, stockholm: Svenska traumaregister.
\url{https://rcsyd.se/swetrau/wp-content/uploads/sites/10/2020/09/A\%CC\%8Arsrapport-SweTrau-2019.pdf}
(accessed 2022).}

\leavevmode\vadjust pre{\hypertarget{ref-Teasdale2014}{}}%
\CSLLeftMargin{16 }
\CSLRightInline{Teasdale G, Maas A, Lecky F, Manley G, Stocchetti N,
Murray G. The glasgow coma scale at 40 years: Standing the test of time.
\emph{The Lancet Neurology} 2014;\textbf{13}:844--54.}

\end{CSLReferences}

\end{document}
